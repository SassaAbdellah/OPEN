\section{Choosing $d$ for a straight Path}
%\index{Straight Path}
In this section, we discuss the choice of d in case of a straigh path.
For a straight path, we have that the deflection is zero, i.e:
\begin{equation}
 \label{straightDeflection}
  h=0
\end{equation}
and that the direct line between two DRPs is equal to the path between these DRPs, i.e:

\begin{equation}
 \label{straightPath}
  \tilde{d}=d
\end{equation}

Such that the formula for the general defect $D$ \label{generalDefectFormula} 
simplifies to
\begin{equation}
 \label{defectStraightPath}
	D:=\pi*{t_r}^2 -(d*2*dist)
\end{equation}

While $F$ becomes
\begin{equation}
  F:=(d*2*dist)
\end{equation}


Giving the following simplified formula for $\frac{D}{F}$:
%\index{Straight Path!Ratio of false positives and false negatives}
\begin{equation}
label{DF_straight_line}
	\frac{D}{F}=\frac{\pi*{t_r}^2 -(d*2*dist)}{d*2*dist} 
\end{equation}

Where the formal parameter $t_r$ might still implicitely contain $d$ !


\subsection{Optimal Choice of d for Straight Line and Testradius $t_{r3}$}
%\index{Straight Path!Optimal Choice of d}
Substituting the $t_{r3}$ as defined in  \label{testradius_tr3},
formula \ref{DF_straight_line} becomes:
\begin{equation}
\label{DF_straight_line_tr3}
\frac{D}{F}=\frac{\pi(\frac{1}{2}d^2+d*dist+dist^2) -(d*2*dist)}{d*2*dist}
\end{equation}

Which becomes minimal for choosing $d$ as:
\begin{equation}
\label{optimal_d_straight_line_tr3}
 d:=(1-dist)+\sqrt{1-2*dist+dist^2-\frac{d*dist^2}{\pi}} 
\end{equation}




